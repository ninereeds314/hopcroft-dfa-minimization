\section{Experimental results and conclusions}

In order to obtain timing information, the algorithm was applied to two classes of finite automata.
Automata in the first class are given by $A(n) = (\{1, 2, \ldots, n\}, \{0, 1\}, \delta, \{ 1 \})$
where $\delta (1,0) = \delta (1,1) = 1$ and
$\delta (i, 0) = i - 1$ and $\delta (i, 1) = i$ for $2 \leq i \leq n$.
Automata in the second class are given (for even $n$) by
$B(n) = (\{1, 2, \ldots, n\}, \{0, 1\}, \delta, \{i|1 \leq i \leq n/2 \})$ where
$\delta (i, 0) = \delta (i, 1) = n/2 + 2i - 1$ and
$\delta (n/4 + i, 0) = \delta (n/4 + i, 1) = 2i - 1$ for $1 \leq i \leq n/4$
and $\delta (n/2+i, 0) = \delta (n/2+i, 1) = 2i - 1$ for
$n/2 < i \leq n$. The running times on an IBM 360/67 for the two classes are listed in Table \ref{results01}.

\begin{table}[ht]
\caption{time in seconds}
\label{results01}
\begin{center}
\begin{tabular}{c|c|c}
$n$     &  $A(n)$              &  $B(n)$             \\
\hline
        &                      &                     \\
$100$   &  $   \frac{37}{60}$  &  $   \frac{2}{3}$   \\
        &                      &                     \\
$1000$  &  $5  \frac{50}{60}$  &  $6  \frac{3}{4}$   \\
        &                      &                     \\
$2003$  &  $11 \frac{38}{60}$  &  $13 \frac{2}{3}$   \\
\end{tabular}
\end{center}
\end{table}

Note that $A(n)$ is the example which required $n^2$ steps for previous algorithms.

Our algorithm is particularly suited for $A(n)$ and a detailed analysis shows that the running time
should grow linearly with the number of states as the experimental evidence indicates. The worst case for
our algorithm is typified by $B(n)$ in which blocks are always partitioned equally. The running time for
$B(n)$ should grow as $n log n$ for both the current algorithm and for previously published algorithms. The
results seem to indicate that the algorithm is practical for minimizing states in finite automata (or testing
equivalence of finite automata) of up to several thousand states.

\section{References}

1. Harrison, M. A., Introduction to Switching and Automata Theory, McGraw-Hill, New York, 1965.

2. McCluskey, E. J., Introduction to the Theory of Switching Circuits, McGraw-Hill, New York, 1965.
