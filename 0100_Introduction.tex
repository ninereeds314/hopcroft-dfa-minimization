\section{Introduction}

Most basic texts on finite automata give algorithms for minimizing the number of states in a finite
automaton. However, a worst case analysis of these algorithms indicate that they are $n^2$ processes
where $n$ is the number of states. For finite automata with large numbers of states, these algorithms
are grossly inefficient. Thus in this paper we describe an algorithm for minimizing the states in
which the asymptotic running time in a worst case analysis grows as $n \log n$. The constant of
proportionality depends linearly on the number of input symbols. Clearly the same algorithm can be
used to determine if two finite automata are equivalent.

The essence of previously published algorithms was to first partition the states according to their
outputs. The blocks of the partitions are then repeatedly refined by examining the successor state on a
given input for each state in the block. States whose successor states on a given input are in different
blocks are placed in separate blocks. When no further refinement is possible, all states in the same block of
the partition can be shown to be equivalent. Consider the example in Figure \ref{example01}.
The initial partition is
$(1,2,3,4,5)(6)$. Since on input $0$, the successor states of states 1, 2, 3 and 4 are in the first
block of the partition and the successor of state 5 is in the second block, the first iteration
refines the partition into the blocks $(1,2,3,4)(5)$ and $(6)$. Successive refinements yield
$(1,2,3)(4)(5)(6)$, $(1,2)(3)(4)(5)(6)$ and $(1)(2)(3)(4)(5)(6)$. Thus in this example all pairs
of states are inequivalent. For this example it is seen that as many as $n$ iterations may be required
and the total number of steps needed to execute the algorithm if implemented in a straightforward
fashion on a digital computer is $n^2$.

\begin{figure}[ht]
\caption{Example 1}
\label{example01}
\begin{center}
\begin{tabular}{c|cc|c}
      & \multicolumn{2}{|c|}{Input} &  \\
State & 0 & 1 & Output \\
\hline
1     & 2 & 1 & 0      \\
2     & 3 & 2 & 0      \\
3     & 4 & 3 & 0      \\
4     & 5 & 4 & 0      \\
5     & 6 & 5 & 0      \\
6     & 6 & 6 & 1      \\
\multicolumn{1}{c}{}  & \multicolumn{2}{c}{Next}  & \multicolumn{1}{c}{} \\
\multicolumn{1}{c}{}  & \multicolumn{2}{c}{State} & \multicolumn{1}{c}{} \\
\end{tabular}
\end{center}
\end{figure}

The algorithm proposed in this paper may also require $n$ iterations but the work per iteration summed
over all iterations yields only $n log n$. We illustrate the algorithm by an example before specifying it in
detail. Extensive use of list processing is employed to reduce the computation time. First the state table
is inverted to obtain the table shown in Figure \ref{inverted02}.
The states are partitioned according to their outputs
$(1,2,3,4,5)(6)$. Next a block and an input symbol on which the partition is refined are selected. Assume
the block $(6)$ and input $0$ are selected. The states in each block are further partitioned depending on
whether on input $0$ their next state is in block $(6)$ or not. Thus the next partition is $(1,2,3,4)(5)(6)$.
Note that had we partitioned on the block $(1,2,3,4,5)$ and input $0$ we would have obtained the same result.

\begin{figure}[ht]
\caption{Inverted Table}
\label{inverted02}
\begin{center}
\begin{tabular}{c|cc|c}
       & \multicolumn{2}{|c|}{Input} &  \\
States & 0 & 1 & Output \\
\hline
1      & -    & 1 & 0      \\
2      & 1    & 2 & 0      \\
3      & 2    & 3 & 0      \\
4      & 3    & 4 & 0      \\
5      & 4    & 5 & 0      \\
6      & 5, 6 & 6 & 1      \\
\multicolumn{1}{c}{}  & \multicolumn{2}{c}{Previous}  & \multicolumn{1}{c}{} \\
\multicolumn{1}{c}{}  & \multicolumn{2}{c}{State} & \multicolumn{1}{c}{} \\
\end{tabular}
\end{center}
\end{figure}

More generally, once we have partitioned on a block
and an input symbol, we need never partition on that
block and input symbol again until the block is
split and then we need only partition on one of
the two subblocks. Since the time needed to
partition on a block is proportional to the
transitions into the block and since we can always
select the half with fewer transitions, the total
number of steps in the algorithm is bounded by
$n log n$.
