\section{Analysis of the running time}

The running time of the algorithm is clearly dependent on the implementation. The algorithm has been
programmed in ALGOL. Since the implementation consists of approximately 300 ALGOL statements we shall simply
indicate how the various steps were implemented and discuss informally their running time.

The sets such as $\delta^{-1} (s, a)$, $L(a)$, etc. were represented by linked lists in such a way that an item
could be added or deleted at the beginning of the list in a fixed number of steps. Vectors were also
maintained to indicate if a state was or was not on a given list. This eliminates searching a list simply
to determine if the item is on the list and is essential in Step 7c. The sets $B(i)$ and $a(i)$ were
represented as doubly linked lists so that an item could be added or deleted anywhere in the list in a fixed
number of steps once the position is given. The structure was such that given a state $s$, its position in
$B(i)$ and $a(i)$ could be determined in a fixed number of steps.

Steps 1, 2, 3 and 4 are executed only once andd require time proportional to the product of the number of
states times the number of input symbols. Steps 5 through 8 form a simple loop. The time necessary to
traverse the loop for given $a$ in $I$ and $i$ in $L(a)$ is proportional to the number of state transitions
on input $a$ terminating on states in $B(i)$. (To see this, note that Steps 5, 6 and 8 are finite. In
Step 7 we do not need to examine $B(j)$ for each $j < k$ to see if there exists a $t$ in $B(j)$ with $\delta (t, a)$
in $a(i)$. Rather we look at each state in $a(i)$ and then consult the inverse state table to find each $t$
such that $\delta (t, a)$ is in $a(i)$. Each time a new $t$ is found, the block containing $t$ is located and
placed on a list of states to be split off from the block. The block is then placed on a list of blocks
which have been refined if it is not already on the list. Finally we go down the list of blocks which have
been refined and actually partition them. The number of blocks we must look at must be less than the number
of state transitions on input $a$ terminating on states in $B(i)$. The time necessary to actually partition
a block is proportional to the number of states to be split off. When the number is summed over all blocks
which are partitioned it adds up to the number of state transitions on input $a$ terminating on states in
$B(i)$.) Let $k$ be the constant of proportionality.

Consider the time spent in the loop of Step 5 through 8 for a given input symbol $a$. Assume that at
step 5 the blocks of the partition are $B(1), B(2), \ldots, B(m)$ and that $L(a) = \{ i_1, i_2, \ldots, i_r \}$ . Let
$\{ i_{r+l}, i_{r+2}, \ldots, i_m \} = \{1, 2, \ldots, m\} - L(a)$. The claim is made
that the total time spent on traversals of the
loop for which input symbol $a$ is selected in Step 5 until the program terminates is bounded by

\begin{displaymath}
T = k(\sum_{j=1}^{r} a_{i_j} \log a_{i_j} + \sum_{j=r+1}^{m} a_{i_j} \log (a_{i_j} / 2))
\end{displaymath}

Clearly the bound is valid if the algorithm has terminated. If the loop is traversed for an input symbol
other than $a$, then the time spent is not included in $T$. However, since blocks get split and the set
$L(a)$ modified we must show that the new value of $T$, call it $\hat{T}$, is less than or equal to the old value
of $T$. If a block whose index is in $L(a)$ is partitioned, then a term of the form $b \log b$ is replaced by
the expression $c \log c + (b - c) \log (b - c)$ which decreases the value of $T$. If a block whose index is not
in $L(a)$ is partitioned, then a term of the form $b \log b/2$ is replaced by the expression

\begin{displaymath}
c \log c + (b - c) \log (b - c) / 2
\end{displaymath}

where $c \leq b-c$. Since $c \leq b/2$ and $(b - c)/2 \leq b/2$,

\begin{displaymath}
\begin{array}{rl}
c \log c + (b - c) \log (b - c) / 2   &   \leq  c \log b/2 + (b - c) \log b/2  \\
                                      &   \leq  b \log b/2                     \\
\end{array}
\end{displaymath}

In either case $\hat{T}$ is less than $T$. Finally, assume $r \neq 0$ and $a$ and some $\vartheta$ in $L(a)$ has been
selected at Step 5. As we showed earlier, the time around the loop is bounded by $ka_{\vartheta}$. Thus by induction,
the total time is bounded by

\begin{displaymath}
k[a_{\vartheta} + a_{\vartheta} \log (a_{\vartheta} / 2) + \sum_{j=1, j \neq \vartheta}^{r} a_{i_j} \log a_{i_j} + \sum_{j=r+1}^{m} a_{i_j} \log (a_{i_j} / 2)]
\end{displaymath}

We must show that this expression is less than or equal to $T$. That is, we need to show

\begin{displaymath}
a_{\vartheta} + a_{\vartheta} \log a_{\vartheta} / 2 \leq a_{\vartheta} \log a_{\vartheta}
\end{displaymath}

Clearly $a_{\vartheta} + a_{\vartheta} \log a_{\vartheta} / 2 = a_{\vartheta} (\log a_{\vartheta} / 2 + \log 2) = a_{\vartheta} \log a_{\vartheta}$.
This completes the proof of the claim.

The first time Step 5 is executed the formula for $T$ is bounded by $kn \log n$. Multiply by the number
of input symbols and adding in the time for Steps 1 through 4 yields a total bound proportional to $n \log n$.
