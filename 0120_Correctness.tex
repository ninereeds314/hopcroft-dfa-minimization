\section{Correctness of the algorithm}

The claim is made that on termination of the algorithm two states are equivalent if and only if they are
in the same block. The algorithm must terminate since the only times that an index is added to $L(a)$ for
some $a$ in $I$ are in Step 4 which is executed only once and in Step 7c. An index is added at Step 7c only
after a refinement of a block of the partition. Each time Step 6 is executed, an index is removed from $L(a)$
for some $a$. Thus the algorithm must terminate.

It is easily shown by induction on the number of times Step 7a is executed that if $s$ is in $B(i)$ and
$t$ is in $B(j)$, $i \neq j$, then $s$ is not equivalent to $t$. Clearly, it is true the first time Step 7a is
executed since only two blocks exist, $B(1)$ containing only final states and $B(2)$ containing only nonfinal
states. Blocks are refined at Step 7a only when successor states on a given input have previously been
shown to be inequivalent.

To see that two inequivalent states cannot be in the same block when the algorithm terminates, assume
that states $s$ and $t$ are in $B(i)$ and that $s$ and $t$ are not equivalent. Without loss of generality,
assume $\delta (s, a)$ is in $B(j)$ and $\delta (t, a)$ is in $B(k)$ where $j \neq k$. (If $\delta (s, a)$ and $\delta (t, a)$ are in the
same block then there exists a shortest $x$ such that $\delta (s, x)$ and $\delta (t, x)$ are in distinct blocks. Clearly
an $x$ exists and hence a shortest $x$ since for some $x$ one or the other of $\delta (s, x)$ and $\delta (t, x)$ but not
both is in a final state and each block consists solely of final or solely of nonfinal states. Let $a$ be the
last symbol of $x$ and write $x = ya$. Then $\delta (s, y)$ and $\delta (t, y)$ are in the same block, $\delta (s, y)$ and $\delta (t, y)$
are not equivalent and $\delta (\delta (s, y), a)$ and $\delta (\delta (t, y), a)$ are in different blocks. Replace $s$ by $\delta (s, y)$ and
replace $t$ by $\delta (t, y)$.) Consider the point at which the block containing $\delta (s, a)$ and $\delta (t, a)$ was
partitioned so that $\delta (s, a)$ and $\delta (t, a)$ first appeared in separate subblocks. At that point one of the
two subblocks was placed in $L(a)$. When this subblock is removed from $L(a)$, the block containing $s$ and
$t$ is partitioned with $s$ and $t$ going into separate subblocks. Thus $s$ and $t$ cannot both be in $B(i)$,
a contradiction.
